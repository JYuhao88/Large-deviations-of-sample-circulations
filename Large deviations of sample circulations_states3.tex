\documentclass[a4paper,11pt]{article}
\usepackage{graphicx,epsfig}
\usepackage{fancyhdr,fancybox,float}
\usepackage[title]{appendix}
\usepackage{indentfirst}
\usepackage{verbatim}
\usepackage[sort&compress, numbers]{natbib}
\usepackage{geometry}
\usepackage{extarrows,chemarrow,xypic}
\usepackage{color}
\usepackage[small]{caption2}
\usepackage{microtype}
\DisableLigatures[f]{encoding = *, family = *}


%%%%%%%%%% 字体支持 %%%%%%%%%%%%
%\usepackage{ccmap}                  % 使pdfLatex生成的文件支持复制等
%\usepackage{CJK,CJKnumb,CJKulem}    % 中文支持
\usepackage{times}                   % 包括 Times Roman + Helvetica + Courier
%\usepackage{palatino}               % 包括 Palatino + Helvetica + Courier
%\usepackage{newcent}                % 包括 New Century Schoolbook + Avant Garde + Courier
%\usepackage{bookman}                % 包括 Bookman + Avant Garde + Courier

%%%%%%%%%% 数学符号公式 %%%%%%%%%%
%\usepackage{latexsym}
%\usepackage{amsmath}                % AMS LaTeX宏包
\usepackage{amssymb}                 % 用来排版漂亮的数学公式
%\usepackage{amsbsy}
\usepackage{amsthm}
%\usepackage{amsfonts}
\usepackage{mathrsfs}                % 英文花体字体
%\usepackage{bm}                     % 数学公式中的黑斜体
\usepackage{bbm}
%\usepackage{relsize}                % 调整公式字体大小:\mathsmaller, \mathlarger
\usepackage{hyperref}                % 生成书签

%字体
\renewcommand{\rmdefault}{ptm}
\renewcommand{\captionlabeldelim}{.}
%字号
\newcommand{\paperfont}{\fontsize{11pt}{1.2\baselineskip}\selectfont}
%页边距
\geometry{top=0.9in,bottom=0.9in,left=1in,right=1in}
%段落缩进
\parindent 4ex
\pagebreak[4]

%%%%%%%%%% 正文 %%%%%%%%%%
\begin{document}
%\linenumbers
	
%%%%%%%%%% 定理类环境的定义 %%%%%%%%%%
\theoremstyle{definition}
\makeatletter
\thm@headfont{\bf}
\makeatother
\newtheorem{theorem}{Theorem}[section]
\newtheorem{definition}[theorem]{Definition}
\newtheorem{lemma}[theorem]{Lemma}
\newtheorem{proposition}[theorem]{Proposition}
\newtheorem{corollary}[theorem]{Corollary}
\newtheorem{remark}[theorem]{Remark}
\newtheorem{example}[theorem]{Example}
\newtheorem{assumption}[theorem]{Assumption}

%%%%%%%%%% 页眉和页脚的设置 %%%%%%%%%%
\lhead{}
\rhead{}
\lfoot{}
\rfoot{}

%%%%%%%%%% 一些重定义 %%%%%%%%%%
\renewcommand{\refname}{References}
\renewcommand{\figurename}{Figure}
\renewcommand{\tablename}{Table}
\renewcommand{\proofname}{Proof}
	
%%%%%%%%%% 符号重定义 %%%%%%%%%%
\newcommand{\diag}{\mathrm{diag}}
\newcommand{\tr}{\mathrm{tr}}
\newcommand{\re}{\mathrm{Re}}
\newcommand{\one}{\mathbbm{1}}
\newcommand{\Pnum}{\mathbb{P}}
\newcommand{\Enum}{\mathbb{E}}
\newcommand{\Rnum}{\mathbb{R}}
\newcommand{\dnum}{\mathrm{d}}
\newcommand{\hyper}{{}_2F_1}
\newcommand{\confl}{{}_1F_1}

\title{The rate function for circulation of three state Markov chain}

\author{Yuhao Jiang}
\date{\zhtoday}

\maketitle
%\tableofcontents
\thispagestyle{empty}

%%%%%%%%%% Page Styles %%%%%%%%%%%
\paperfont

%%%%%%%%%% Abstract %%%%%%%%%%
\begin{abstract}
\end{abstract}

\section{The rate function for circulation of three state Markov chain}
\begin{align*}
    I(\nu) 
    &= \sum_{i \in I} \left(-\nu^{i} \log \frac{\nu^{i}}{w^{i}} + (\nu^{i} - \nu_{i}) \log(\nu^{i} - \nu_{i})\right) \\
    &-(\tilde{\nu} - \sum_{i \in I} \nu_i) \log (\tilde{\nu} - \sum_{i \in I} \nu_i)
    +\sum_{t \in C_{\infty}} \nu_t \log \nu_t \\
    &- (\nu_{1} \log w_{1} + \nu_{2} \log w_{2} + \nu_{3} \log w_{3})\\
    &- (\nu_{12} + \nu_{123}) \log(w_{12} + w_{123}) \\
    &- (\nu_{13} + \nu_{132}) \log(w_{13} + w_{132}) \\
    &- (\nu_{12} + \nu_{132}) \log(w_{12} + w_{132}) \\
    &- (\nu_{23} + \nu_{123}) \log(w_{23} + w_{123}) \\
    &- (\nu_{13} + \nu_{123}) \log(w_{13} + w_{123}) \\
    &- (\nu_{23} + \nu_{132}) \log(w_{23} + w_{132}) 
\end{align*}
where $I=\{1, 2, 3\}$ is the state space for Markov chain。
$$\mathcal{C}_{\infty} = \{(1), (2), (3), (1,2), (2,3), (1,3), (1,2,3), (1, 32)$$
is the class of all cycles occurring.
$\nu_c$ is the frequence of $c$ occurring, $w_c$ is the circulation of $c$.\\
And $\nu^{i} = \sum_{J_{c_s}(i)=1} \nu_{c_s}$, such as $\nu^{1} = \nu_{1} + \nu_{12} + \nu_{13} + \nu_{123} +\nu_{132}$。
$\tilde{\nu} = \nu_{1} + \nu_{2} + \nu_{3} + \nu_{12} + \nu_{13} + \nu_{23} + \nu_{123} +\nu_{132}$。\\
$w^i$, $w_i$ has the similar definition.

Refer to the result of $p_{ii}=0$, we can get:
\begin{align*}
    I(\nu)
    &= 
    \nu_1 \log(\frac{\nu_1}{w_1} / \frac{\nu^1}{w^1})
    + \nu_2 \log(\frac{\nu_2}{w_2} / \frac{\nu^2}{w^2})
    + \nu_3 \log(\frac{\nu_3}{w_3} / \frac{\nu^3}{w^3}) \\
    &+\nu_{12} \log(\frac{1}{p_{12}p_{21}} \frac{\nu_{12}}{\nu_{12}+\nu_{23}+\nu_{13}+\nu_{123}+\nu_{132}} 
    \frac{\nu^{1}-\nu_{1}}{\nu^{1}}\frac{\nu^{2}-\nu_{2}}{\nu^{2}}) \\
    &+ \nu_{13} \log(\frac{1}{p_{13}p_{31}} \frac{\nu_{13}}{\nu_{12}+\nu_{23}+\nu_{13}+\nu_{123}+\nu_{132}}
    \frac{\nu^{1}-\nu_{1}}{\nu^{1}}\frac{\nu^{3}-\nu_{3}}{\nu^{3}}) \\
    &+ \nu_{23} \log(\frac{1}{p_{23}p_{32}} \frac{\nu_{23}}{\nu_{12}+\nu_{23}+\nu_{13}+\nu_{123}+\nu_{132}} 
    \frac{\nu^{3}-\nu_{3}}{\nu^{3}}\frac{\nu^{2}-\nu_{2}}{\nu^{2}}) \\
    &+ \nu_{123} \log(\frac{1}{p_{12}p_{23}p_{31}} \frac{\nu_{123}}{\nu_{12}+\nu_{23}+\nu_{13}+\nu_{123}+\nu_{132}}
    \frac{\nu^{1}-\nu_{1}}{\nu^{1}}\frac{\nu^{2}-\nu_{2}}{\nu^{2}} \frac{\nu^{3}-\nu_{3}}{\nu^{3}})) \\
    &+ \nu_{132} \log(\frac{1}{p_{13}p_{32}p_{21}} \frac{\nu_{132}}{\nu_{12}+\nu_{23}+\nu_{13}+\nu_{123}+\nu_{132}}
    \frac{\nu^{1}-\nu_{1}}{\nu^{1}}\frac{\nu^{2}-\nu_{2}}{\nu^{2}} \frac{\nu^{3}-\nu_{3}}{\nu^{3}})) \\
\end{align*}

So we need to valid following proposition. if $\nu=w$, then
$$
\frac{1}{p_{12}p_{21}} \frac{\nu_{12}}{\nu_{12}+\nu_{23}+\nu_{13}+\nu_{123}+\nu_{132}} \frac{\nu^{1}-\nu_{1}}{\nu^{1}}\frac{\nu^{2}-\nu_{2}}{\nu^{2}} = 1
$$
$$
\frac{1}{p_{12}p_{23}p_{31}} \frac{\nu_{123}}{\nu_{12}+\nu_{23}+\nu_{13}+\nu_{123}+\nu_{132}}\frac{\nu^{1}-\nu_{1}}{\nu^{1}}\frac{\nu^{2}-\nu_{2}}{\nu^{2}} \frac{\nu^{3}-\nu_{3}}{\nu^{3}} = 1
$$
We use
\begin{align*}
    \frac{w^{1}-w_{1}}{w^{1}} &= \frac{w_{12}+w_{13}+w_{123}+w_{132}}{w_{1}+w_{12}+w_{13}+w_{123}+w_{132}} \\
    &= p_{12}+p_{13} \\
    &= p_{21}+p_{31} \\
    &= 1 - p_{11} \\
    &= D(\{2, 3\}^c)
\end{align*}
and
\begin{align*}
    w_{12} &= p_{12}p_{21} \frac{D(\{1, 2\}^c)}{\sum_{i\in I} D(\{i\}^c)} \\
    w_{13} &= p_{13}p_{31} \frac{D(\{1, 3\}^c)}{\sum_{i\in I} D(\{i\}^c)} \\
    w_{23} &= p_{23}p_{32} \frac{D(\{2, 3\}^c)}{\sum_{i\in I} D(\{i\}^c)} \\
    w_{123} &= p_{12}p_{23}p_{31} \frac{D(\{1, 2, 3\}^c)}{\sum_{i\in I} D(\{i\}^c)} \\
    w_{132} &= p_{13}p_{32}p_{21} \frac{D(\{1, 2, 3\}^c)}{\sum_{i\in I} D(\{i\}^c)}
\end{align*}
substitute $w$, the above proposition can be valid.

By above validation, we also know:
\begin{align*}
    w_{12}+w_{13}+w_{23}+w_{123}+w_{132} &= \frac{(1-p_{11})(1-p_{22})(1-p_{33})}{\sum_{i\in I} D(\{i\}^c)} \\
    &= \frac{\Pi_{i, j} D(\{i, j\}^c)}{\sum_{i\in I} D(\{i\}^c)}
\end{align*}


\section{$p_{13}=0$}
rate function:
\begin{align*}
    I(\nu)
    &= -(\nu^{1} \log (\frac{\nu^{1}}{w^{1}}) + \nu^{2} \log (\frac{\nu^{2}}{w^{2}}) + \nu^{3} \log (\frac{\nu^{3}}{w^{3}})) \\
    &+ \nu_{1} \log (\frac{\nu_{1}}{w_{1}}) + \nu_{2} \log (\frac{\nu_{2}}{w_{2}}) + \nu_{3} \log (\frac{\nu_{3}}{w_{3}}) \\
    &+ (\nu_{12}+\nu_{123}) \log(\frac{\nu_{12}+\nu_{123}}{w_{12}+w{123}}) +
    (\nu_{23}+\nu_{123}) \log(\frac{\nu_{23}+\nu_{123}}{w_{23}+w{123}}) \\
    &+ \nu_{12} \log(\frac{\nu_{12}}{w_{12}}) + \nu_{23} \log(\frac{\nu_{23}}{w_{23}}) + \nu_{123} \log(\frac{\nu_{123}}{w_{123}}) \\
\end{align*}
That is
\begin{align*}
    I(\nu) &=  \sum_{i, j \in I} \left(\sum_{c \in \mathcal{C_{\infty}}, J_c(i, j)=1}
    \nu_c \right) \log(\frac{\sum_{c \in \mathcal{C_{\infty}}, J_c(i, j)=1} w_c \sum_{c \in \mathcal{C_{\infty}}, J_c(i)=1} \nu_c}{\sum_{c \in \mathcal{C_{\infty}}, J_c(i, j)=1} \nu_c \sum_{c \in \mathcal{C_{\infty}}, J_c(i)=1} w_c}) \\
    &= \sum_{i, j \in I} \left(\sum_{c \in \mathcal{C_{\infty}}, J_c(i, j)=1}
    \nu_c \right) \log(\frac{\sum_{c \in \mathcal{C_{\infty}}, J_c(i, j)=1} w_c }{\sum_{c \in \mathcal{C_{\infty}}, J_c(i, j)=1} \nu_c }
    /\frac{\sum_{c \in \mathcal{C_{\infty}}, J_c(i)=1} w_c}{\sum_{c \in \mathcal{C_{\infty}}, J_c(i)=1} \nu_c}) \\
\end{align*}

\section{$p_{ii}=0$}
rate function:
\begin{align*}
    I(\nu)
    &= \sum_{i \in I} (\nu^{i}-\nu_{i}) \log (w^{i}-w_{i})
    - (\tilde{\nu} - \sum_{i \in I} \nu_{i}) \log (\tilde{\nu} - \sum_{i \in I} \nu_{i}) \\
    &+ \nu_{12} \log \nu_{12} + \nu_{23} \log \nu_{23} + \nu_{13} \log \nu_{13} +\nu_{123} \log \nu_{123} + \nu_{132} \log \nu_{132} \\
    % &- (\nu_{1} \log w_{1} + \nu_{2} \log w_{2} + \nu_{3} \log w_{3})\\
    &- (\nu_{12} + \nu_{123}) \log(w_{12} + w_{123}) \\
    &- (\nu_{13} + \nu_{132}) \log(w_{13} + w_{132}) \\
    &- (\nu_{12} + \nu_{132}) \log(w_{12} + w_{132}) \\
    &- (\nu_{23} + \nu_{123}) \log(w_{23} + w_{123}) \\
    &- (\nu_{13} + \nu_{123}) \log(w_{13} + w_{123}) \\
    &- (\nu_{23} + \nu_{132}) \log(w_{23} + w_{132}) 
\end{align*}
We can simply it to following formula:
\begin{align*}
    I(\nu)
    &= \nu_{12} \log(\frac{w_{12} + w_{13} +w_{123} + w_{132}}{w_{12}+w_{123}}
                    \frac{w_{12} + w_{23} +w_{123} + w_{132}}{w_{12}+w_{132}}\\
    &               \frac{\nu_{12}}{\nu_{12}+\nu_{23}+\nu_{13}+\nu_{123}+\nu_{132}}) \\
    &+ \nu_{13} \log(\frac{w_{12} + w_{23} +w_{123} + w_{132}}{w_{13}+w_{123}}
                    \frac{w_{13} + w_{23} +w_{123} + w_{132}}{w_{13}+w_{132}}\\
    &               \frac{\nu_{13}}{\nu_{12}+\nu_{23}+\nu_{13}+\nu_{123}+\nu_{132}}) \\
    &+ \nu_{23} \log(\frac{w_{12} + w_{23} +w_{123} + w_{132}}{w_{23}+w_{123}}
                    \frac{w_{13} + w_{23} +w_{123} + w_{132}}{w_{23}+w_{132}}\\
    &               \frac{\nu_{23}}{\nu_{12}+\nu_{23}+\nu_{13}+\nu_{123}+\nu_{132}}) \\
    &+ \nu_{123} \log(\frac{w_{12} + w_{13} +w_{123} + w_{132}}{w_{12}+w_{123}}
                    \frac{w_{12} + w_{23} +w_{123} + w_{132}}{w_{23}+w_{123}}\\
    &               \frac{w_{13} + w_{23} +w_{123} + w_{132}}{w_{13}+w_{132}}
                    \frac{\nu_{123}}{\nu_{12}+\nu_{23}+\nu_{13}+\nu_{123}+\nu_{132}}) \\
    &+ \nu_{132} \log(\frac{w_{12} + w_{23} +w_{123} + w_{132}}{w_{13}+w_{123}}
                    \frac{w_{13} + w_{23} +w_{123} + w_{132}}{w_{23}+w_{132}}\\
    &               \frac{w_{12} + w_{23} +w_{123} + w_{132}}{w_{12}+w_{132}}
                    \frac{\nu_{132}}{\nu_{12}+\nu_{23}+\nu_{13}+\nu_{123}+\nu_{132}}) \\
    &= \nu_{12} \log(\frac{1}{p_{12}p_{21}} \frac{\nu_{12}}{\nu_{12}+\nu_{23}+\nu_{13}+\nu_{123}+\nu_{132}}) \\
    &+ \nu_{13} \log(\frac{1}{p_{13}p_{31}} \frac{\nu_{13}}{\nu_{12}+\nu_{23}+\nu_{13}+\nu_{123}+\nu_{132}}) \\
    &+ \nu_{23} \log(\frac{1}{p_{23}p_{32}} \frac{\nu_{23}}{\nu_{12}+\nu_{23}+\nu_{13}+\nu_{123}+\nu_{132}}) \\
    &+ \nu_{123} \log(\frac{1}{p_{12}p_{23}p_{31}} \frac{\nu_{123}}{\nu_{12}+\nu_{23}+\nu_{13}+\nu_{123}+\nu_{132}}) \\
    &+ \nu_{132} \log(\frac{1}{p_{13}p_{32}p_{21}} \frac{\nu_{132}}{\nu_{12}+\nu_{23}+\nu_{13}+\nu_{123}+\nu_{132}})
\end{align*}

Associationg the expression of circulation, and we can further considered
\begin{align*}
    w_{12} &= p_{12}p_{21} \frac{D({1, 2}^c)}{\sum_{i\in I} D(\{i\}^c)} \\
    w_{13} &= p_{13}p_{31} \frac{D({1, 3}^c)}{\sum_{i\in I} D(\{i\}^c)} \\
    w_{23} &= p_{23}p_{32} \frac{D({2, 3}^c)}{\sum_{i\in I} D(\{i\}^c)} \\
    w_{123} &= p_{12}p_{23}p_{31} \frac{D({1, 2, 3}^c)}{\sum_{i\in I} D(\{i\}^c)} \\
    w_{132} &= p_{13}p_{32}p_{21} \frac{D({1, 2, 3}^c)}{\sum_{i\in I} D(\{i\}^c)} \\
\end{align*}

We only need to valid :
$$
w_{12} + w_{23} + w_{13} + w_{123} + w_{132} = 1 / \sum_{i\in I} D(\{i\}^c)
$$
Using $p$ to express $w$, we can get ($p_{ii}=0$):
$$
p_{12}p_{21} + p_{13}p_{31} + p_{23}p_{32} + p_{12}p_{23}p_{31} + p_{13}p_{32}p_{21} = 1
$$
And we can valid it by calculation
\end{document}