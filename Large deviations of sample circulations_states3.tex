\documentclass[cn,hazy,egreen,14pt,normal]{elegantnote}
\title{三状态环流的rate function}

\author{姜瑜浩}
\date{\zhtoday}

%command
\usepackage{array}
\usepackage{amssymb}
\usepackage{graphicx,float,subfigure}
\usepackage{subfigure}
\usepackage[T1]{fontenc}
\usepackage[utf8]{inputenc}
\usepackage{mathtools}   % loads »amsmath«

\newcommand{\ccr}[1]{\makecell{{\color{#1}\rule{1cm}{1cm}}}}
\newcommand{\grad}{\text{grad}~}

\begin{document}
\section{三状态环流大偏差的rate function:}
三状态环流大偏差的rate function:
\begin{align*}
    I(\nu) 
    &= \sum_{i \in I} \left(-\nu^{i} \log \frac{\nu^{i}}{w^{i}} + (\nu^{i} - \nu_{i}) \log(\nu^{i} - \nu_{i})\right) \\
    &-(\tilde{\nu} - \sum_{i \in I} \nu_i) \log (\tilde{\nu} - \sum_{i \in I} \nu_i)
    +\sum_{t \in C_{\infty}} \nu_t \log \nu_t \\
    &- (\nu_{1} \log w_{1} + \nu_{2} \log w_{2} + \nu_{3} \log w_{3})\\
    &- (\nu_{12} + \nu_{123}) \log(w_{12} + w_{123}) \\
    &- (\nu_{13} + \nu_{132}) \log(w_{13} + w_{132}) \\
    &- (\nu_{12} + \nu_{132}) \log(w_{12} + w_{132}) \\
    &- (\nu_{23} + \nu_{123}) \log(w_{23} + w_{123}) \\
    &- (\nu_{13} + \nu_{123}) \log(w_{13} + w_{123}) \\
    &- (\nu_{23} + \nu_{132}) \log(w_{23} + w_{132}) 
\end{align*}
其中 $I=\{1, 2, 3\}$ 是马氏链的状态空间。
$$\mathcal{C}_{\infty} = \{(1), (2), (3), (1,2), (2,3), (1,3), (1,2,3), (1, 32)$$
表示所有可能出现环的集合。
$\nu_c$ 表示环 $c$ 出现的频率,$w_c$ 表示环 $c$ 的环流。\\
且 $\nu^{i} = \sum_{J_{c_s}(i)=1} \nu_{c_s}$,例如 $\nu^{1} = \nu_{1} + \nu_{12} + \nu_{13} + \nu_{123} +\nu_{132}$。
$\tilde{\nu} = \nu_{1} + \nu_{2} + \nu_{3} + \nu_{12} + \nu_{13} + \nu_{23} + \nu_{123} +\nu_{132}$。\\
$w^i$, $w_i$ 表示类似的含义。

由$p_{ii}=0$情况的想法,可以得:
\begin{align*}
    I(\nu)
    &= 
    \nu_1 \log(\frac{\nu_1}{w_1} / \frac{\nu^1}{w^1})
    + \nu_2 \log(\frac{\nu_2}{w_2} / \frac{\nu^2}{w^2})
    + \nu_3 \log(\frac{\nu_3}{w_3} / \frac{\nu^3}{w^3}) \\
    &+\nu_{12} \log(\frac{1}{p_{12}p_{21}} \frac{\nu_{12}}{\nu_{12}+\nu_{23}+\nu_{13}+\nu_{123}+\nu_{132}} 
    \frac{\nu^{1}-\nu_{1}}{\nu^{1}}\frac{\nu^{2}-\nu_{2}}{\nu^{2}}) \\
    &+ \nu_{13} \log(\frac{1}{p_{13}p_{31}} \frac{\nu_{13}}{\nu_{12}+\nu_{23}+\nu_{13}+\nu_{123}+\nu_{132}}
    \frac{\nu^{1}-\nu_{1}}{\nu^{1}}\frac{\nu^{3}-\nu_{3}}{\nu^{3}}) \\
    &+ \nu_{23} \log(\frac{1}{p_{23}p_{32}} \frac{\nu_{23}}{\nu_{12}+\nu_{23}+\nu_{13}+\nu_{123}+\nu_{132}} 
    \frac{\nu^{3}-\nu_{3}}{\nu^{3}}\frac{\nu^{2}-\nu_{2}}{\nu^{2}}) \\
    &+ \nu_{123} \log(\frac{1}{p_{12}p_{23}p_{31}} \frac{\nu_{123}}{\nu_{12}+\nu_{23}+\nu_{13}+\nu_{123}+\nu_{132}}
    \frac{\nu^{1}-\nu_{1}}{\nu^{1}}\frac{\nu^{2}-\nu_{2}}{\nu^{2}} \frac{\nu^{3}-\nu_{3}}{\nu^{3}})) \\
    &+ \nu_{132} \log(\frac{1}{p_{13}p_{32}p_{21}} \frac{\nu_{132}}{\nu_{12}+\nu_{23}+\nu_{13}+\nu_{123}+\nu_{132}}
    \frac{\nu^{1}-\nu_{1}}{\nu^{1}}\frac{\nu^{2}-\nu_{2}}{\nu^{2}} \frac{\nu^{3}-\nu_{3}}{\nu^{3}})) \\
\end{align*}
数值上已经验证:
当 $\nu=w$时,有:
$$
\frac{1}{p_{12}p_{21}} \frac{\nu_{12}}{\nu_{12}+\nu_{23}+\nu_{13}+\nu_{123}+\nu_{132}} \frac{\nu^{1}-\nu_{1}}{\nu^{1}}\frac{\nu^{2}-\nu_{2}}{\nu^{2}} = 1
$$
$$
\frac{1}{p_{12}p_{23}p_{31}} \frac{\nu_{123}}{\nu_{12}+\nu_{23}+\nu_{13}+\nu_{123}+\nu_{132}}\frac{\nu^{1}-\nu_{1}}{\nu^{1}}\frac{\nu^{2}-\nu_{2}}{\nu^{2}} \frac{\nu^{3}-\nu_{3}}{\nu^{3}} = 1
$$
\section{$p_{13}=0$的情形}
rate function:
\begin{align*}
    I(\nu)
    &= -(\nu^{1} \log (\frac{\nu^{1}}{w^{1}}) + \nu^{2} \log (\frac{\nu^{2}}{w^{2}}) + \nu^{3} \log (\frac{\nu^{3}}{w^{3}})) \\
    &+ \nu_{1} \log (\frac{\nu_{1}}{w_{1}}) + \nu_{2} \log (\frac{\nu_{2}}{w_{2}}) + \nu_{3} \log (\frac{\nu_{3}}{w_{3}}) \\
    &+ (\nu_{12}+\nu_{123}) \log(\frac{\nu_{12}+\nu_{123}}{w_{12}+w{123}}) +
    (\nu_{23}+\nu_{123}) \log(\frac{\nu_{23}+\nu_{123}}{w_{23}+w{123}}) \\
    &+ \nu_{12} \log(\frac{\nu_{12}}{w_{12}}) + \nu_{23} \log(\frac{\nu_{23}}{w_{23}}) + \nu_{123} \log(\frac{\nu_{123}}{w_{123}}) \\
\end{align*}
即
\begin{align*}
    I(\nu) &=  \sum_{i, j \in I} \left(\sum_{c \in \mathcal{C_{\infty}}, J_c(i, j)=1}
    \nu_c \right) \log(\frac{\sum_{c \in \mathcal{C_{\infty}}, J_c(i, j)=1} w_c \sum_{c \in \mathcal{C_{\infty}}, J_c(i)=1} \nu_c}{\sum_{c \in \mathcal{C_{\infty}}, J_c(i, j)=1} \nu_c \sum_{c \in \mathcal{C_{\infty}}, J_c(i)=1} w_c}) \\
    &= \sum_{i, j \in I} \left(\sum_{c \in \mathcal{C_{\infty}}, J_c(i, j)=1}
    \nu_c \right) \log(\frac{\sum_{c \in \mathcal{C_{\infty}}, J_c(i, j)=1} w_c }{\sum_{c \in \mathcal{C_{\infty}}, J_c(i, j)=1} \nu_c }
    /\frac{\sum_{c \in \mathcal{C_{\infty}}, J_c(i)=1} w_c}{\sum_{c \in \mathcal{C_{\infty}}, J_c(i)=1} \nu_c}) \\
\end{align*}

\section{$p_{ii}=0$的情形}
rate function:
\begin{align*}
    I(\nu)
    &= \sum_{i \in I} (\nu^{i}-\nu_{i}) \log (w^{i}-w_{i})
    - (\tilde{\nu} - \sum_{i \in I} \nu_{i}) \log (\tilde{\nu} - \sum_{i \in I} \nu_{i}) \\
    &+ \nu_{12} \log \nu_{12} + \nu_{23} \log \nu_{23} + \nu_{13} \log \nu_{13} +\nu_{123} \log \nu_{123} + \nu_{132} \log \nu_{132} \\
    % &- (\nu_{1} \log w_{1} + \nu_{2} \log w_{2} + \nu_{3} \log w_{3})\\
    &- (\nu_{12} + \nu_{123}) \log(w_{12} + w_{123}) \\
    &- (\nu_{13} + \nu_{132}) \log(w_{13} + w_{132}) \\
    &- (\nu_{12} + \nu_{132}) \log(w_{12} + w_{132}) \\
    &- (\nu_{23} + \nu_{123}) \log(w_{23} + w_{123}) \\
    &- (\nu_{13} + \nu_{123}) \log(w_{13} + w_{123}) \\
    &- (\nu_{23} + \nu_{132}) \log(w_{23} + w_{132}) 
\end{align*}
化简得:
\begin{align*}
    I(\nu)
    &= \nu_{12} \log(\frac{w_{12} + w_{13} +w_{123} + w_{132}}{w_{12}+w_{123}}
                    \frac{w_{12} + w_{23} +w_{123} + w_{132}}{w_{12}+w_{132}}\\
    &               \frac{\nu_{12}}{\nu_{12}+\nu_{23}+\nu_{13}+\nu_{123}+\nu_{132}}) \\
    &+ \nu_{13} \log(\frac{w_{12} + w_{23} +w_{123} + w_{132}}{w_{13}+w_{123}}
                    \frac{w_{13} + w_{23} +w_{123} + w_{132}}{w_{13}+w_{132}}\\
    &               \frac{\nu_{13}}{\nu_{12}+\nu_{23}+\nu_{13}+\nu_{123}+\nu_{132}}) \\
    &+ \nu_{23} \log(\frac{w_{12} + w_{23} +w_{123} + w_{132}}{w_{23}+w_{123}}
                    \frac{w_{13} + w_{23} +w_{123} + w_{132}}{w_{23}+w_{132}}\\
    &               \frac{\nu_{23}}{\nu_{12}+\nu_{23}+\nu_{13}+\nu_{123}+\nu_{132}}) \\
    &+ \nu_{123} \log(\frac{w_{12} + w_{13} +w_{123} + w_{132}}{w_{12}+w_{123}}
                    \frac{w_{12} + w_{23} +w_{123} + w_{132}}{w_{23}+w_{123}}\\
    &               \frac{w_{13} + w_{23} +w_{123} + w_{132}}{w_{13}+w_{132}}
                    \frac{\nu_{123}}{\nu_{12}+\nu_{23}+\nu_{13}+\nu_{123}+\nu_{132}}) \\
    &+ \nu_{132} \log(\frac{w_{12} + w_{23} +w_{123} + w_{132}}{w_{13}+w_{123}}
                    \frac{w_{13} + w_{23} +w_{123} + w_{132}}{w_{23}+w_{132}}\\
    &               \frac{w_{12} + w_{23} +w_{123} + w_{132}}{w_{12}+w_{132}}
                    \frac{\nu_{132}}{\nu_{12}+\nu_{23}+\nu_{13}+\nu_{123}+\nu_{132}}) \\
    &= \nu_{12} \log(\frac{1}{p_{12}p_{21}} \frac{\nu_{12}}{\nu_{12}+\nu_{23}+\nu_{13}+\nu_{123}+\nu_{132}}) \\
    &+ \nu_{13} \log(\frac{1}{p_{13}p_{31}} \frac{\nu_{13}}{\nu_{12}+\nu_{23}+\nu_{13}+\nu_{123}+\nu_{132}}) \\
    &+ \nu_{23} \log(\frac{1}{p_{23}p_{32}} \frac{\nu_{23}}{\nu_{12}+\nu_{23}+\nu_{13}+\nu_{123}+\nu_{132}}) \\
    &+ \nu_{123} \log(\frac{1}{p_{12}p_{23}p_{31}} \frac{\nu_{123}}{\nu_{12}+\nu_{23}+\nu_{13}+\nu_{123}+\nu_{132}}) \\
    &+ \nu_{132} \log(\frac{1}{p_{13}p_{32}p_{21}} \frac{\nu_{132}}{\nu_{12}+\nu_{23}+\nu_{13}+\nu_{123}+\nu_{132}})
\end{align*}
可进一步考虑联系环流的表达式
\begin{align*}
    w_{12} &= p_{12}p_{21} \frac{D({1, 2}^c)}{\sum_{i\in I} D(\{i\}^c)} \\
    w_{13} &= p_{13}p_{31} \frac{D({1, 3}^c)}{\sum_{i\in I} D(\{i\}^c)} \\
    w_{23} &= p_{23}p_{32} \frac{D({2, 3}^c)}{\sum_{i\in I} D(\{i\}^c)} \\
    w_{123} &= p_{12}p_{23}p_{31} \frac{D({1, 2, 3}^c)}{\sum_{i\in I} D(\{i\}^c)} \\
    w_{132} &= p_{13}p_{32}p_{21} \frac{D({1, 2, 3}^c)}{\sum_{i\in I} D(\{i\}^c)} \\
\end{align*}

只需证 :
$$
\nu_{12} + \nu_{23} + \nu_{13} + \nu_{123} + \nu_{132} = 1 / \sum_{i\in I} D(\{i\}^c)
$$
即证($p_{ii}=0$):
$$
p_{12}p_{21} + p_{13}p_{31} + p_{23}p_{32} + p_{12}p_{23}p_{31} + p_{13}p_{32}p_{21} = 1
$$
用 $p_{12}, p_{21}, p_{31}$带换其他变量,可证。
\end{document}