% Full chain: pdflatex -> bibtex -> pdflatex -> pdflatex
\documentclass[11pt,en,cite=authoryear]{elegantpaper}

\title{Circulation theory of enzyme kinetics}

\author{Yuhao Jiang}
\date{\today}

%command
\usepackage{array}
\usepackage{amssymb}
\usepackage{graphicx,float,subfigure}
\usepackage{subfigure}
\usepackage[T1]{fontenc}
\usepackage[utf8]{inputenc}
\usepackage{mathtools}   % loads »amsmath«


\begin{document}

\maketitle
We consider the following n-step($n\ge 2$) enzyme kinetics model:
\begin{align} \label{eq:model}
    E + S \rightleftharpoons
    ES \rightleftharpoons
    EP_1 \rightleftharpoons
    EP_2 \rightleftharpoons
    \cdots
    EP_{n-2} \rightleftharpoons
    E + P
\end{align}

where E is an enzyme turning the substrate S into the product P. From the  perspective of a single enzyme molecule, this enzyme kinetics can be modeled as n-step Markov chain $(\xi_l)_{l\ge 0}$, with finite state space S defined on some space $(\Omega, \mathcal{F}, P)$. 

If $n=2$, then this Markov chain only have two state $E$ and $ES$, and we can say the state space $S=\{1, 2\}$, which represent $\{E, ES\}$.

\begin{definition}
    Let $\mathbb{Z}$ be the set of integers, and a periodic function $f$ which maps $\mathbb{Z}$ to S is called circuit function. If $s$ is the smallest positive integer which satisfied $f(n+s) = f(n)$ for each $n \in \mathbb{Z}$, then we called it the period of $f$.
\end{definition}

\begin{definition}
    Two circuit functions $f$ and $g$ in S are called equivalent if there exists some $m \in \mathbb{Z}$ such that $g(n) = f(n+m)$ for each $n \in \mathbb{Z}$.
\end{definition}


According to the above definition, we know $(1, 2, 3), (3, 1, 2)$ and $(2, 3, 1)$ represent the same cycle. So a cycle is also a equivalence class on the space of all
circuit function under the equivalence relation.

\begin{definition}
    Let $\mathcal{C}_n(\omega)$ be the class of cycles occurring along the sample path $(\xi_l)_{l\ge 0}$ until time $n$. Then we use $\mathcal{C}_{\infty}$ to represent the limit of $\mathcal{C}_n$ as $n \rightarrow \infty$. This convergence has been proofed in [].
\end{definition}



\begin{definition} %% follow Jia2016
    Let $k_{c, n}$ denote the number of time that cycle c is formed by a Markov chain up to time $n$. Then the sample circulation $J_{c, n}$ along cycle $c$ by time $t$ is defined as
    $$
    w_{c, n} = \frac{1}{n} k_{c, n}
    $$
    and the circulation $J^c$ along cycle $c$ is a nonnegative real number defined as the following almost sure limit:
    $$
    w_c = \lim_{n \rightarrow \infty} w_{c, n}, \quad a.s.
    $$
    which represents the number of times that cycle $c$ ic formed per unit time.
\end{definition}


Consider the enzyme kinetics model, if the state space $S=\{1, 2, \dots, n\}$, then $\mathcal{C}_{\infty}$ contains  folloing $2n+2$ cycles. 
\begin{itemize}
    \item $n$ simple cycles
    \item $n$ two-state cycles
    \item two n-state cycles
\end{itemize}
Let $|c|$ denotes the length of cycle $c$, and $\nu_c$ is the frequency of cycle $c$ occuring, then $\nu_c \in [0, 1/|c|]$. 
% For example, consider the cycle $c=\{2, 3, 4\}$, then $|c|=3$, and $\nu_c \in [0, \frac{1}{3}]$
\begin{definition}
    We say the law of $(w_{c_1}, w_{c_2}, \dots, w_{c_r})$ satisfies a large deviation principle if:
    \begin{align}
        \lim_{n \rightarrow \infty} \frac{1}{n} \log \mathbb{P}(w_{c, n} = \nu_c, c \in \mathcal{C}_{\infty}) = - \min_{\nu} I(\nu)
    \end{align}
    where $|c| \nu_c = 1$, I: $\Pi_{c} [0, \frac{1}{|c|}] \rightarrow [0, \infty]$.
\end{definition}



\section{Large deviation of circulation for finite Markov chains}


\subsection{Large deviation of circulation for two state Markov chains}
\begin{align*}
    I(\nu) &= \nu_{1}\log(\frac{\nu_{1}}{\nu_{1}+\nu_{12}}/\frac{w_{1}}{w_{1}+w_{12}}) + \nu_{2}\log(\frac{w_{2}}{w_{1}+w_{12}}) + 2\nu_{12} \log (\frac{\nu_{12}}{\nu_{1}+\nu_{12}}/\frac{w_{12}}{w_{1}+w_{12}})
\end{align*}

\subsection{Large deviation of circulation for three state Markov chains}
\begin{align*}
    I(\nu) 
    &= \sum_{i \in I} \left(-\nu^{i} \log \frac{\nu^{i}}{w^{i}} + (\nu^{i} - \nu_{i}) \log(\nu^{i} - \nu_{i})\right) \\
    &-(\tilde{\nu} - \sum_{i \in I} \nu_i) \log (\tilde{\nu} - \sum_{i \in I} \nu_i)
    +\sum_{t \in C_{\infty}} \nu_t \log \nu_t \\
    &- (\nu_{1} \log w_{1} + \nu_{2} \log w_{2} + \nu_{3} \log w_{3})\\
    &- (\nu_{12} + \nu_{123}) \log(w_{12} + w_{123}) \\
    &- (\nu_{13} + \nu_{132}) \log(w_{13} + w_{132}) \\
    &- (\nu_{12} + \nu_{132}) \log(w_{12} + w_{132}) \\
    &- (\nu_{23} + \nu_{123}) \log(w_{23} + w_{123}) \\
    &- (\nu_{13} + \nu_{123}) \log(w_{13} + w_{123}) \\
    &- (\nu_{23} + \nu_{132}) \log(w_{23} + w_{132}) 
\end{align*}
where $I=\{1, 2, 3\}$ is the state space for Markov chains。
$$\mathcal{C}_{\infty} = \{(1), (2), (3), (1,2), (2,3), (1,3), (1,2,3), (1,3,2)$$
is the class of all cycles occurring.
$\nu_c$ is the frequence of $c$ occurring, $w_c$ is the circulation of $c$.\\
And $\nu^{i} = \sum_{J_{c_s}(i)=1} \nu_{c_s}$, such as $\nu^{1} = \nu_{1} + \nu_{12} + \nu_{13} + \nu_{123} +\nu_{132}$。
$\tilde{\nu} = \nu_{1} + \nu_{2} + \nu_{3} + \nu_{12} + \nu_{13} + \nu_{23} + \nu_{123} +\nu_{132}$。\\
$w^i$, $w_i$ has the similar definition.

\subsection{Large deviation of circulation for multi-state Markov chains}



\section{Appendix}
Consider the first n-step of the above Markov chains $(\xi_l)$, we assume that $\xi_{n}=\xi_1$, and number of the cycle $c_i$ occurring is $k_{c_i}$ for each $c_i \in \mathcal{C}_{\infty}$. 
If mark each occurence of $(s, t)$ in $\xi_1, \dots, \xi_n$ by drawing an arrow from $s$ to $t$, we can obtain an oriented graph $G(k)$, and use $\mathcal{E} (G(k))$ to denote the number of Euler circuits on $G(k)$.
\begin{theorem}
    For the graph $G(k)$ induced by the Markov chains $(\xi_l)_{l\ge 0}$, if the size of state space is $s$, then we have the following formula:
    \begin{equation*}
        \mathcal{E} (G(k)) = 
        \binom{k_{12}+k_{14}+k_{+}+k_{-}}{k_{12}, k_{14}, k_{+}, k_{-}} \\
        
    \end{equation*}
\end{theorem}
\end{document}