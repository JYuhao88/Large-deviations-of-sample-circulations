% Full chain: pdflatex -> bibtex -> pdflatex -> pdflatex
\documentclass[11pt,en,cite=authoryear]{elegantpaper}

\title{Circulation theory of enzyme kinetics}

\author{Yuhao Jiang}
\date{\today}

%command
\usepackage{array}
\usepackage{amssymb}
\usepackage{graphicx,float,subfigure}
\usepackage{subfigure}
\usepackage[T1]{fontenc}
\usepackage[utf8]{inputenc}
\usepackage{mathtools}   % loads »amsmath«


\begin{document}

\maketitle

Here we adopt the presentation given by Kalpazatidou.
Let $X=(X_n)_{n\ge 0}$ be a discrete-time Markov chain with finite state space S defined on some space $(\Omega, \mathcal{F}, P)$. \\

\begin{definition}
    A circuit function in a finite set S is a periodic function c, which maps the set $\mathbf{Z}$ of integers to S.
\end{definition}

% \begin{definition}
%     The two circuit function $c$ and $c'$ are equivalent if and only if it exists $k$ such that $c(n) = c'(n+k)$ for each $n \in \mathbf{Z}$, the smallest integer $k$ is the period of $c$. 
% \end{definition}

\begin{definition}
    Let $\mathcal{C}_n(\omega)$ be the class of all cycles occurring along the sample path $\{\xi_l(\omega)\}$
\end{definition}

% \begin{definition}
%     $\Omega_n = \mathcal{S}^{n} = \left\{\omega = (\omega_{n+1}):  \omega_{n+1} = \omega_1 \right\}, n \in Z^+$
%     is $n$ step orbit space, and start from $\omega_1$, along this trajectory, pass $n$ step, can return to $\omega_1$.
% \end{definition}

% \begin{definition}
%     Let
%     $$
%     K_n = \left\{\omega \in \Omega_n : w_{c, n}(\omega) = k^c \right\} 
%     $$
%     denote the orbit space given the number of each circuit, and the orbit length is $n$. And let $|K_n|$ denote the size of $|K_n|$.
%     $h^i = \sum_{J_c(i)=1} k^c$
% \end{definition}

\begin{definition} %% follow Jia2016
    Let $N_n^c$ denote the number of time that cycle c is formed by a Markov chain up to time $n$. Then the sample circulation $J_n^c$ along cycle $c$ by time $t$ is defined as
    $$
    J_n^c = \frac{1}{n} N_n^c
    $$
    and the circulation $J^c$ along cycle $c$ is a nonnegative real number defined as the following almost sure limit:
    $$
    J^c = \lim_{n \rightarrow \infty} J_n^c, \quad a.s.
    $$
    which represents the number of times that cycle $c$ ic formed per unit time.
\end{definition}

\begin{lemma}
    \begin{align*}
        \mathbb{P}(J^c = \nu_c, \forall c \in \mathcal{C}_n(\omega)) = - \min I(\nu)
    \end{align*}
    where 
\end{lemma}

\section{Large deviation of circulation for two state Markov chains}
\begin{align*}
    I(\nu) &= \nu_{1}\log(\frac{\nu_{1}}{\nu_{1}+\nu_{12}}/\frac{w_{1}}{w_{1}+w_{12}}) + \nu_{2}\log(\frac{w_{2}}{w_{1}+w_{12}}) + 2\nu_{12} \log (\frac{\nu_{12}}{\nu_{1}+\nu_{12}}/\frac{w_{12}}{w_{1}+w_{12}})
\end{align*}

\section{Large deviation of circulation for three state Markov chains}
\begin{align*}
    I(\nu) 
    &= \sum_{i \in I} \left(-\nu^{i} \log \frac{\nu^{i}}{w^{i}} + (\nu^{i} - \nu_{i}) \log(\nu^{i} - \nu_{i})\right) \\
    &-(\tilde{\nu} - \sum_{i \in I} \nu_i) \log (\tilde{\nu} - \sum_{i \in I} \nu_i)
    +\sum_{t \in C_{\infty}} \nu_t \log \nu_t \\
    &- (\nu_{1} \log w_{1} + \nu_{2} \log w_{2} + \nu_{3} \log w_{3})\\
    &- (\nu_{12} + \nu_{123}) \log(w_{12} + w_{123}) \\
    &- (\nu_{13} + \nu_{132}) \log(w_{13} + w_{132}) \\
    &- (\nu_{12} + \nu_{132}) \log(w_{12} + w_{132}) \\
    &- (\nu_{23} + \nu_{123}) \log(w_{23} + w_{123}) \\
    &- (\nu_{13} + \nu_{123}) \log(w_{13} + w_{123}) \\
    &- (\nu_{23} + \nu_{132}) \log(w_{23} + w_{132}) 
\end{align*}
where $I=\{1, 2, 3\}$ is the state space for Markov chains。
$$\mathcal{C}_{\infty} = \{(1), (2), (3), (1,2), (2,3), (1,3), (1,2,3), (1,3,2)$$
is the class of all cycles occurring.
$\nu_c$ is the frequence of $c$ occurring, $w_c$ is the circulation of $c$.\\
And $\nu^{i} = \sum_{J_{c_s}(i)=1} \nu_{c_s}$, such as $\nu^{1} = \nu_{1} + \nu_{12} + \nu_{13} + \nu_{123} +\nu_{132}$。
$\tilde{\nu} = \nu_{1} + \nu_{2} + \nu_{3} + \nu_{12} + \nu_{13} + \nu_{23} + \nu_{123} +\nu_{132}$。\\
$w^i$, $w_i$ has the similar definition.

\section{Large deviation of circulation for multi-state Markov chains}

% \begin{thebibliography}{99}
%     \bibitem{mla}
%     KALPAZIDOU, S. L. (1995). "Cycle Representations of Markov Processes" Springer, New York. MR1336140
% \end{thebibliography}

\end{document}