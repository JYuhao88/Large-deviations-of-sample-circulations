\documentclass[11pt,en,cite=authoryear]{elegantpaper}

\title{Circulation theory of enzyme kinetics}

\author{Yuhao Jiang}
\date{\today}

%command
\usepackage{array}
\usepackage{amssymb}
\usepackage{graphicx,float,subfigure}
\usepackage{subfigure}
\usepackage[T1]{fontenc}
\usepackage[utf8]{inputenc}
\usepackage{mathtools}   % loads »amsmath«


\begin{document}

\maketitle

Here we adopt the presentation given by Kalpazatidou.
Let $(\xi_l)_{l\ge 0}$ be a discrete-time Markov chain with finite state space S defined on some space $(\Omega, \mathcal{F}, P)$. \\

% \begin{definition}
%     The two circuit function $c$ and $c'$ are equivalent if and only if it exists $k$ such that $c(n) = c'(n+k)$ for each $n \in \mathbf{Z}$, the smallest integer $k$ is the period of $c$. 
% \end{definition}


% \begin{definition}
%     $\Omega_n = \mathcal{S}^{n} = \left\{\omega = (\omega_{n+1}):  \omega_{n+1} = \omega_1 \right\}, n \in Z^+$
%     is $n$ step orbit space, and start from $\omega_1$, along this trajectory, pass $n$ step, can return to $\omega_1$.
% \end{definition}

% \begin{definition}
%     Let
%     $$
%     K_n = \left\{\omega \in \Omega_n : w_{c, n}(\omega) = k^c \right\} 
%     $$
%     denote the orbit space given the number of each circuit, and the orbit length is $n$. And let $|K_n|$ denote the size of $|K_n|$.
%     $h^i = \sum_{J_c(i)=1} k^c$
% \end{definition}

% \begin{thebibliography}{99}
%     \bibitem{mla}
%     KALPAZIDOU, S. L. (1995). "Cycle Representations of Markov Processes" Springer, New York. MR1336140
% \end{thebibliography}

\begin{definition}
    Let $\{\mu_n: n \in N^*\}$ be a family of probability measures on a Polish
    space $E$. Then we say that $\{\mu_n: n \in N^*\}$ satisfies a large deviation principle with rate $n$ and good rate function $I: E \rightarrow [0, \infty]$ if: 

    (i) for each $\alpha >0$, the level set $\{x\in E: I(x)<\alpha\}$ is compact in $E$;

    (ii) for each closed subset $F$ of $E$,
    \begin{align}
        \limsup_{n \rightarrow \infty} \frac{1}{n} \log \mu_n(F) \leq -\inf_{x\in F} I(x);
    \end{align}

    (iii) for each open subset $U$ of E,
    \begin{align}
        \liminf_{n \rightarrow \infty} \frac{1}{n} \log \mu_n(U) \geq -\inf_{x\in U} I(x);
    \end{align}
\end{definition}

\end{document}