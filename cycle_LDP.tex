\documentclass[cn,hazy,egreen,14pt,normal]{elegantnote}
\title{大偏差理论}

\author{姜瑜浩}
\date{\zhtoday}

%command
\usepackage{array}
\usepackage{amssymb}
\usepackage{graphicx,float,subfigure}
\usepackage{subfigure}
\usepackage[T1]{fontenc}
\usepackage[utf8]{inputenc}
\usepackage{mathtools}   % loads »amsmath«

\newcommand{\ccr}[1]{\makecell{{\color{#1}\rule{1cm}{1cm}}}}
\newcommand{\grad}{\text{grad}~}

\begin{document}
\section*{状态空间大小为3的马氏链的环流的大偏差}
在该问题中有下面8种环流:
$c_1: 1 \rightarrow 1$, $c_2: 2 \rightarrow 2$, $c_3: 3 \rightarrow 3$ \\
$c_4: 1 \rightarrow 2 \rightarrow 1$, $c_5: 1 \rightarrow 3 \rightarrow 1$, $c_6: 2 \rightarrow 3 \rightarrow 2$ \\
$c_7: 1 \rightarrow 2 \rightarrow 3 \rightarrow 1$, $c_8: 1 \rightarrow 3 \rightarrow 2 \rightarrow 1$\\
在有3个状态的图中,考虑所有的 $n$ 步欧拉回路。为了方便,令起始位置的状态为1。 $n$步的欧拉回路可以分解为上述的 8 个环路。而且给定环路的完成顺序可以唯一确定一个欧拉回路。\\
%%令 $k_i$ 表示$n$ 步欧拉回路中,环流 $c_i$ 出现的次数。
$xi$ 表示一条 $n$ 步欧拉回路,且在这条回路中,环流 $c_i$ 出现 $k_i$ 次。
由于这条回路可以唯一分解为上述环流,则每种边出现的次数可以得到,则
$$
k_1 + k_2 + k_3 + 2(k_{12} + k_{13} +k_{23}) + 3(k_{123} + k_{132}) = n 
$$
$$
\mathbb{P}(\xi) = p_{11}^{k_1} p_{22}^{k_2} p_{33}^{k_3} p_{12}^{k_{12}+k_{123}} p_{21}^{k_{12}+k_{132}} p_{13}^{k_{13}+k_{132}} p_{31}^{k_{13}+k_{123}} p_{23}^{k_{23}+k_{123}} p_{32}^{k_{23}+k_{132}} 
$$
只需计算 $A$ 这种欧拉回路有多少种。
由初始状态是1,考虑$c_1, c_4, c_5, c_7, c_8$ (状态1开始,状态1结束)。
对于这几种环流,只有当一个结束,才会有另一个开始。也就是说在$c_7$开始,但没完成之前,不会先完成$c_5$。
由这几种环流的不同排列,可以有
$$
\binom{k_1 + k_{12} + k_{13} + k_{123} +k_{132}}{k_1, ~k_{12}, ~k_{13}, ~k_{123}, ~k_{132}} 
$$
种不同的可能。
先把环流 $c_6$ 嵌入进去,其中 $c_6$可以嵌入在状态2,也可以嵌入在状态3上,所以会有这些种可能:
$$
\binom{k_{12} +k_{13} + k_{123} +k_{132} + k_{23} - 1}{k_{23}}
$$
再把环流 $c_2$ 和 $c_3$ 嵌入在上述可能的欧拉路径中,依次有这些中嵌入方法 \\
$c_2$ : $\binom{k_{12} + k_{123} + k_{132} +k_{23} +k_2 -1}{k_2}$, \\
$c_3$ : $\binom{k_{13} +k_{123} +k_{132} +k_{23} +k_3 -1}{k_3}$。\\

不能先插入 $2->2$ 和 $3->3$ ,再插入 $2->3->2$
比如在环 $1->3->2->1$ 中先插入 $2->2$,即使规定在$1->3->2->1$中,不能在 2 位置插入 $2->3->2$,但是得到的 $1->3->2->2->1$的第二个2中,插入 $2->3->2$,也会产生类似的效果。并且我们无法准确清楚有多少 $2->2$ 已经插入在环 $1->3->2->1$上。


以状态1开始,有下面多种情况:
\begin{align*}
    A_1 &= \\ 
    &\binom{k_1 + k_{12} + k_{13} + k_{123} +k_{132}}{k_1, ~k_{12}, ~k_{13}, ~k_{123}, ~k_{132}} 
    \binom{k_{12} +k_{13} + k_{123} +k_{132} + k_{23} - 1}{k_{23}} \\
    &\binom{k_{12} + k_{123} + k_{132} +k_{23} +k_2 -1}{k_2} \binom{k_{13} +k_{123} +k_{132} +k_{23} +k_3 -1}{k_3}
\end{align*}

以状态2开始,有下面多种情况:
\begin{align*}
    A_2 &= \\ 
    &\binom{k_2 + k_{12} + k_{23} + k_{123} +k_{132}}{k_2, ~k_{12}, ~k_{23}, ~k_{123}, ~k_{132}} 
    \binom{k_{12} +k_{23} + k_{123} +k_{132} + k_{13} - 1}{k_{13}} \\
    &\binom{k_{12} + k_{123} + k_{132} +k_{13} +k_1 -1}{k_1} \binom{k_{23} +k_{123} +k_{132} +k_{13} +k_3 -1}{k_3}
\end{align*}

以状态3开始,有下面多种情况:
\begin{align*}
    A_3 &= \\ 
    &\binom{k_3 + k_{13} + k_{23} + k_{123} +k_{132}}{k_3, ~k_{13}, ~k_{23}, ~k_{123}, ~k_{132}} 
    \binom{k_{13} +k_{23} + k_{123} +k_{132} + k_{12} - 1}{k_{12}} \\
    &\binom{k_{13} + k_{123} + k_{132} +k_{12} +k_1 -1}{k_1} \binom{k_{23} +k_{123} +k_{132} +k_{12} +k_2 -1}{k_2}
\end{align*}

\begin{align*}
    A_1 &= \\ 
    &\binom{k_{12} + k_{13} + k_{123} +k_{132}}{k_{12}, ~k_{13}, ~k_{123}, ~k_{132}} \\
    & \binom{k_1 + k_{12} + k_{13} + k_{123} +k_{132}}{k_1}
    \binom{k_{12} +k_{13} + k_{123} +k_{132} + k_{23} - 1}{k_{23}} \\
    &\binom{k_{12} + k_{123} + k_{132} +k_{23} +k_2 -1}{k_2} \binom{k_{13} +k_{123} +k_{132} +k_{23} +k_3 -1}{k_3}
\end{align*}
由 
\begin{align*}
    \frac{1}{n} \log C_{n-1}^m &=  \frac{1}{n} \log \frac{n(n-1)(n-2)\cdots (m+1)}{m(m-1)\cdots 1}, \quad n\rightarrow \infty\\
    &= \frac{1}{n} \log \frac{(n-1)(n-2)\cdots (n-m)}{m(m-1)\cdots 1} , \quad n\rightarrow \infty\\
    &= \frac{1}{n} \log \frac{n(n-1)\cdots (n-m+1)}{m(m-1)\cdots 1} + O(\frac{\log n}{n}), \quad n\rightarrow \infty\\
    &= \frac{1}{n} \log C_n^m + O\left(\frac{\log n}{n}\right), \quad n\rightarrow \infty
\end{align*}
和 stirling 公式:
\begin{align*}
    \frac{1}{n} A_1 &= \\ 
    &\frac{1}{n} \binom{k_{12} + k_{13} + k_{123} +k_{132}}{k_{12}, ~k_{13}, ~k_{123}, ~k_{132}} \\
    & \binom{k_1 + k_{12} + k_{13} + k_{123} +k_{132}}{k_1}
    \binom{k_{12} +k_{13} + k_{123} +k_{132} + k_{23}}{k_{23}} \\
    &\binom{k_2 + k_{12} + k_{123} + k_{132} +k_{23}}{k_2} \binom{k_3 + k_{13} +k_{123} +k_{132} +k_{23}}{k_3} + O(\frac{\log n}{n}) \\
    &= h(\nu_{12} + \nu_{13} + \nu_{123} +\nu_{132}) - \left[h(\nu_{12}) + h(\nu_{13}) + h(\nu_{123}) + h(\nu_{132})\right] \\
    &+ h(\nu_1 + \nu_{12} + \nu_{13} + \nu_{123} +\nu_{132}) - \left[h(\nu_1) + h(\nu_{12} + \nu_{13} + \nu_{123} +\nu_{132})\right] \\
    &+ h(\nu_{12} + \nu_{13} + \nu_{23}+\nu_{123}+\nu_{132}) - \left[h(\nu_{23}) + h(\nu_{12} + \nu_{13}+\nu_{123}+\nu_{132})\right] \\
    &+ h(\nu_2 + \nu_{12} + \nu_{123} + \nu_{132} +\nu_{23}) - \left[h(\nu_2) + h(\nu_{12} + \nu_{123} + \nu_{132} +\nu_{23})\right] \\
    &+ h(\nu_3 + \nu_{13} +\nu_{123} +\nu_{132} +\nu_{23}) - \left[h(\nu_3) + h(\nu_{13} +\nu_{123} +\nu_{132} +\nu_{23})\right] \\
    &= h(\nu_{12} + \nu_{13} + \nu_{23}+\nu_{123}+\nu_{132}) \\
    &+ h(\nu_1 + \nu_{12} + \nu_{13} + \nu_{123} +\nu_{132}) \\
    &+ h(\nu_2 + \nu_{12} + \nu_{123} + \nu_{132} +\nu_{23}) \\
    &+ h(\nu_3 + \nu_{13} +\nu_{123} +\nu_{132} +\nu_{23}) \\
    &- \left[h(\nu_1) + h(\nu_2) + h(\nu_3) + h(\nu_{12}) + h(\nu_{13}) + h(\nu_{23}) + h(\nu_{123}) + h(\nu_{132})\right] \\
    &- \biggl(h(\nu_{12} + \nu_{13} + \nu_{123} +\nu_{132}) + h(\nu_{12} + \nu_{123} + \nu_{132} +\nu_{23}) \\
    & + h(\nu_{13} +\nu_{123} +\nu_{132} +\nu_{23})\biggr)
\end{align*}
$$
\mathbb{P}(\xi) = p_{11}^{k_1} p_{22}^{k_2} p_{33}^{k_3} p_{12}^{k_{12}+k_{123}} p_{21}^{k_{12}+k_{132}} p_{13}^{k_{13}+k_{132}} p_{31}^{k_{13}+k_{123}} p_{23}^{k_{23}+k_{123}} p_{23}^{k_{23}+k_{132}} 
$$
环流分解
\begin{align*}
    \pi_1 p_{11} &= w_{1} \\
    \pi_2 p_{22} &= w_{2} \\
    \pi_3 p_{33} &= w_{3} \\
    \pi_1 p_{12} &= w_{12} + w_{123}\\
    \pi_1 p_{13} &= w_{13} + w_{132}\\
    \pi_2 p_{21} &= w_{12} + w_{123}\\
    \pi_2 p_{23} &= w_{23} + w_{132}\\
    \pi_3 p_{31} &= w_{13} + w_{123}\\
    \pi_3 p_{23} &= w_{23} + w_{132}
\end{align*}
可得
\begin{equation*}
    \left\{
        \begin{array}{l}
            (w_{12}+w_{123})p_{11} - w_1 p_{12} = 0 \\
            (w_{13}+w_{132})p_{11} - w_1 p_{13} = 0 \\
            p_{11} + p_{12} + p_{13} = 1
        \end{array}
    \right.
\end{equation*}

\begin{align*}
    p_{11} &= \frac{w_{1}}{w_{1}+w_{12}+w_{13}+w_{123}+w_{132}} \\
    p_{12} &= \frac{w_{12} + w_{123}}{w_{1}+w_{12}+w_{13}+w_{123}+w_{132}}\\
    p_{13} &= \frac{w_{13} + w_{132}}{w_{1}+w_{12}+w_{13}+w_{123}+w_{132}}\\
    p_{22} &= \frac{w_{2}}{w_{2}+w_{12}+w_{23}+w_{123}+w_{132}} \\
    p_{21} &= \frac{w_{12} + w_{132}}{w_{2}+w_{12}+w_{23}+w_{123}+w_{132}}\\
    p_{23} &= \frac{w_{23} + w_{123}}{w_{2}+w_{12}+w_{23}+w_{123}+w_{132}}\\
    p_{33} &= \frac{w_{3}}{w_{3}+w_{23}+w_{13}+w_{123}+w_{132}} \\
    p_{31} &= \frac{w_{13} + w_{123}}{w_{3}+w_{23}+w_{13}+w_{123}+w_{132}}\\
    p_{32} &= \frac{w_{23} + w_{132}}{w_{3}+w_{23}+w_{13}+w_{123}+w_{132}}
\end{align*}

\begin{align*}
    \frac{1}{n} \log \mathbb{P}(\xi)
    =& \frac{1}{n} \log p_{11}^{k_1} p_{22}^{k_2} p_{33}^{k_3} p_{12}^{k_{12}+k_{123}} p_{21}^{k_{12}+k_{132}} p_{13}^{k_{13}+k_{132}} p_{31}^{k_{13}+k_{123}} p_{23}^{k_{23}+k_{123}} p_{23}^{k_{23}+k_{132}} \\
    =& \nu_{1} \log w_{1} + \nu_{2} \log w_{2} + \nu_{3} \log w_{3} \\
    &+ (\nu_{12} + \nu_{123}) \log(w_{12} + w_{123}) \\
    &+ (\nu_{13} + \nu_{132}) \log(w_{13} + w_{132}) \\
    &+ (\nu_{12} + \nu_{132}) \log(w_{12} + w_{132}) \\
    &+ (\nu_{23} + \nu_{123}) \log(w_{23} + w_{123}) \\
    &+ (\nu_{13} + \nu_{123}) \log(w_{13} + w_{123}) \\
    &+ (\nu_{23} + \nu_{132}) \log(w_{23} + w_{132}) \\
    &- (\nu_1 + \nu_{12} + \nu_{13} + \nu_{123} + \nu_{132}) \log (w_1 + w_{12} + w_{13} + w_{123} + w_{132}) \\
    &- (\nu_2 + \nu_{12} + \nu_{23} + \nu_{123} + \nu_{132}) \log (w_2 + w_{12} + w_{23} + w_{123} + w_{132}) \\
    &- (\nu_3 + \nu_{13} + \nu_{23} + \nu_{123} + \nu_{132}) \log (w_3 + w_{13} + w_{23} + w_{123} + w_{132}) \\
\end{align*}

% rate function 会不会是?
% \begin{align*}
%     &I(\nu_{1}, \nu_{2}, \nu_{3}, \nu_{12}, \nu_{13}, \nu_{23}, \nu_{123}, \nu_{132}) =  \nu_{1} \log \frac{\nu_{1}}{w_{1}} + \nu_{2} \log \frac{\nu_{2}}{w_{2}} + \nu_{3} \log \frac{\nu_{3}}{w_{3}} \\
%     &+(\nu_{12} + \nu_{123}) \log(\frac{\nu_{12} + \nu_{123}}{w_{12} + w_{123}}) \\
%     &+ (\nu_{13} + \nu_{132}) \log(\frac{\nu_{13} + \nu_{132}}{w_{13} + w_{132}}) \\
%     &+ (\nu_{12} + \nu_{132}) \log(\frac{\nu_{12} + \nu_{132}}{w_{12} + w_{132}}) \\
%     &+ (\nu_{23} + \nu_{123}) \log(\frac{\nu_{23} + \nu_{123}}{w_{23} + w_{123}}) \\
%     &+ (\nu_{13} + \nu_{123}) \log(\frac{\nu_{13} + \nu_{123}}{w_{13} + w_{123}}) \\
%     &+ (\nu_{23} + \nu_{132}) \log(\frac{\nu_{23} + \nu_{132}}{w_{23} + w_{132}}) \\
%     &- (\nu_1 + \nu_{12} + \nu_{13} + \nu_{123} + \nu_{132}) \log (\frac{\nu_1 + \nu_{12} + \nu_{13} + \nu_{123} + \nu_{132}}{w_1 + w_{12} + w_{13} + w_{123} + w_{132}}) \\
%     &- (\nu_2 + \nu_{12} + \nu_{23} + \nu_{123} + \nu_{132}) \log (\frac{\nu_2 + \nu_{12} + \nu_{23} + \nu_{123} + \nu_{132}}{w_2 + w_{12} + w_{23} + w_{123} + w_{132}}) \\
%     &- (\nu_3 + \nu_{13} + \nu_{23} + \nu_{123} + \nu_{132}) \log (\frac{\nu_3 + \nu_{13} + \nu_{23} + \nu_{123} + \nu_{132}}{w_3 + w_{13} + w_{23} + w_{123} + w_{132}}) \\
% \end{align*}

% 是否会有这样公式?
% \begin{align*}
%     A_1 &= \\ 
%     &e^{O(log(n))} \binom{k_{1}+k_{12}+k_{13}+k_{123}+k_{132}}{k_{1}, ~k_{12}+k_{123}, ~k_{13}+k_{132}} 
%     \binom{k_{2} +k_{23} + k_{12} +k_{123} + k_{132}}{k_{2} ,~k_{12} + k_{132}, ~k_{23} + k_{123}} \\
%     &\binom{k_{3} + k_{13} + k_{123} +k_{23} +k_{132}}{k_{3} ,~k_{13} + k_{123} ~k_{23} +k_{132}} 
% \end{align*}
% $A_1$ 表示从状态出发,在n步的回路中,使得每种环按照规定频数出现的路径数量。

% 对于n状态的情况:
% \begin{align*}
%     I(\nu) &=  \sum_{i, j \in I} \left(\sum_{c \in \mathcal{C_{\infty}}, J_c(i, j)=1}
%     \nu_c \right) \log(\frac{\sum_{c \in \mathcal{C_{\infty}}, J_c(i, j)=1} w_c \sum_{c \in \mathcal{C_{\infty}}, J_c(i)=1} \nu_c}{\sum_{c \in \mathcal{C_{\infty}}, J_c(i, j)=1} \nu_c \sum_{c \in \mathcal{C_{\infty}}, J_c(i)=1} w_c}) \\
%     &= \sum_{i, j \in I} \left(\sum_{c \in \mathcal{C_{\infty}}, J_c(i, j)=1}
%     \nu_c \right) \log(\frac{\sum_{c \in \mathcal{C_{\infty}}, J_c(i, j)=1} w_c }{\sum_{c \in \mathcal{C_{\infty}}, J_c(i, j)=1} \nu_c }
%     /\frac{\sum_{c \in \mathcal{C_{\infty}}, J_c(i)=1} w_c}{\sum_{c \in \mathcal{C_{\infty}}, J_c(i)=1} \nu_c}) \\
% \end{align*}

令 $\nu^{i} = \frac{k_i}{n}, ~i\in I$,即 $\nu^{i} = \sum_{J_{c_s}(i)=1} \nu_{c_s}$.
$\tilde{\nu} = \sum_{c \in \mathcal{C}_{\infty}} \nu_{c}$.
其中 $h(x) = x \log x$
则:  
\begin{align*}
    \frac{1}{n} \log \mathbb{P} (J^{c_s} &= \frac{k_s}{n}, s = 1, 2, \dots 8) = \\
    &\sum_{i \in I} \left(\nu^{i} \log \nu^{i} - (\nu^{i} - \nu_{(i)}) \log(\nu^{i} - \nu_{(i)})\right) \\
    &+(\tilde{\nu} - \sum_{i \in I} \nu_{i}) \log (\tilde{\nu} - \sum_{i \in I} \nu_{i})
    -\sum_{t \in C_{\infty}} \nu_t \log \nu_t \\
    &+ \mathbb{P}(\xi)  + O(\frac{\log n}{n}) 
\end{align*}

rate function:
\begin{align*}
    I(\nu) 
    &= \sum_{i \in I} \left(-\nu^{i} \log \frac{\nu^{i}}{w^{i}} + (\nu^{i} - \nu_{(i)}) \log(\nu^{i} - \nu_{(i)})\right) \\
    &-(\tilde{\nu} - \sum_{i \in I} \nu_i) \log (\tilde{\nu} - \sum_{i \in I} \nu_i)
    +\sum_{t \in C_{\infty}} \nu_t \log \nu_t \\
    &- (\nu_{1} \log w_{1} + \nu_{2} \log w_{2} + \nu_{3} \log w_{3})\\
    &- (\nu_{12} + \nu_{123}) \log(w_{12} + w_{123}) \\
    &- (\nu_{13} + \nu_{132}) \log(w_{13} + w_{132}) \\
    &- (\nu_{12} + \nu_{132}) \log(w_{12} + w_{132}) \\
    &- (\nu_{23} + \nu_{123}) \log(w_{23} + w_{123}) \\
    &- (\nu_{13} + \nu_{123}) \log(w_{13} + w_{123}) \\
    &- (\nu_{23} + \nu_{132}) \log(w_{23} + w_{132}) 
    + O(\frac{\log n}{n}) 
\end{align*}

\section*{状态空间大小为4的马氏链的环流的大偏差}
在所有的回路中使得,初始状态为1,环流 $c_t$有$k_t$的路径数量:
\begin{align*}
    &A_1 = \\
    &\binom{k_{12}+k_{14}+k_{1234}+k_{1432}}{k_{12}, k_{14}, k_{1234}, k_{1432}} \\
    &\binom{k_{1}+k_{12}+k_{14}+k_{1234}+k_{1432}}{k_{12}+k_{14}+k_{1234}+k_{1432}}
    \binom{k_{2}+k_{12}+k_{23}+k_{1234}+k_{1432}-1}{k_{12}+k_{23}+k_{1234}+k_{1432}-1}\\
    &\binom{k_{3}+k_{23}+k_{34}+k_{1234}+k_{1432}-1}{k_{23}+k_{34}+k_{1234}+k_{1432}-1}
    \binom{k_{4}+k_{14}+k_{34}+k_{1234}+k_{1432}-1}{k_{14}+k_{34}+k_{1234}+k_{1432}-1}\\
    &\biggl(
        \sum_{k_{23}^{12}+k_{23}^{14}=k_{23}} \sum_{k_{34}^{12}+k_{34}^{14}=k_{34}}
        \binom{k_{23}^{12}+k_{12}+k_{1234}-1}{k_{23}^{12}}
        \binom{k_{34}^{12}+k_{1234}+k_{23}^{12}-1}{k_{34}^{12}} \\
        &\binom{k_{34}^{14}+k_{14}+k_{1432}-1}{k_{34}^{14}}
        \binom{k_{23}^{14}+k_{34}^{14}+k_{1432}-1}{k_{23}^{14}} 
    \biggr)
\end{align*}
其中 $k_{23}^{12}$ 和 $k_{34}^{12}$ 分别表示 $k_{23}$,$k_{34}$ 嵌入到 $k_{1234}$ 和 $k_{12}$ (从1出发,第二步为2的环)的数量。$k_{23}^{14}$ 和 $k_{34}^{14}$ 分别表示 $k_{23}$,$k_{34}$ 嵌入到 $k_{1432}$ 和 $k_{14}$ (从1出发,第二步为4的环)的数量。($k_{23}^{12}, k_{23}^{14}, k_{34}^{12}, k_{34}^{14} \geq  0$)

由马氏链的常返性,对于足够大的 $n$,路径中必然包含所有状态,则:
$$
\frac{1}{n} A_j \leq A_i \leq n A_j
$$
故:
$$
\frac{1}{n} \sqrt[4]{A_1, A_2, A_3, A_4} \leq A_i \leq n \sqrt[4]{A_1, A_2, A_3, A_4}
$$

从而
\begin{align}
    \frac{1}{n} \log A_1 
    = \frac{1}{4n} \log (A_1, A_2, A_3, A_4) + O(\frac{\log n}{n})
\end{align}

\begin{align*}
    \frac{1}{n} \log \mathbb{P}(\xi)
    =& \sum_{i \in I} \left(\nu^{i} \log p_{ii} - (\nu^{i} - \nu_{(i)}) \log w_i \right) \\
    &+ (\nu_{12} + \nu_{1234}) \log(w_{12} + w_{1234}) \\
    &+ (\nu_{23} + \nu_{1234}) \log(w_{23} + w_{1234}) \\
    &+ (\nu_{34} + \nu_{1234}) \log(w_{34} + w_{1234}) \\
    &+ (\nu_{41} + \nu_{1234}) \log(w_{41} + w_{1234}) \\
    &+ (\nu_{14} + \nu_{1432}) \log(w_{14} + w_{1432}) \\
    &+ (\nu_{43} + \nu_{1432}) \log(w_{43} + w_{1432}) \\
    &+ (\nu_{32} + \nu_{1432}) \log(w_{32} + w_{1432}) \\
    &+ (\nu_{21} + \nu_{1432}) \log(w_{21} + w_{1432}) 
    + O(\frac{\log n}{n}) 
\end{align*}

% 则在所有回路中使得环流 $c_t$有$k_t$的路径数量:
% \begin{align*}
%     &\binom{k_{12}+k_{14}+k_{1234}+k_{1432}}{k_{12}, k_{14}, k_{1234}, k_{1432}} \\
%     &\binom{k_{1}+k_{12}+k_{14}+k_{1234}+k_{1432}}{k_{12}+k_{14}+k_{1234}+k_{1432}}
%     \binom{k_{2}+k_{12}+k_{23}+k_{1234}+k_{1432}-1}{k_{12}+k_{23}+k_{1234}+k_{1432}-1}\\
%     &\binom{k_{3}+k_{23}+k_{34}+k_{1234}+k_{1432}-1}{k_{23}+k_{34}+k_{1234}+k_{1432}-1}
%     \binom{k_{4}+k_{14}+k_{34}+k_{1234}+k_{1432}-1}{k_{23}+k_{34}+k_{1234}+k_{1432}-1}\\
%     &\biggl(
%         \sum_{k_{23}^{12}+k_{23}^{14}=k_{23}} \sum_{k_{34}^{12}+k_{34}^{14}=k_{34}}
%         \binom{k_{23}^{12}+k_{12}+k_{1234}-1}{k_{23}^{12}}
%         \binom{k_{34}^{12}+k_{1234}+k_{23}^{12}-1}{k_{34}^{12}} \\
%         &\binom{k_{34}^{14}+k_{14}+k_{1432}-1}{k_{34}^{14}}
%         \binom{k_{23}^{14}+k_{34}^{14}+k_{1432}-1}{k_{23}^{14}} 
%     \biggr) + \\
%     &\binom{k_{12}+k_{23}+k_{1234}+k_{1432}}{k_{12}, k_{23}, k_{1234}, k_{1432}} \\
%     &\binom{k_{1}+k_{12}+k_{14}+k_{1234}+k_{1432}-1}{k_{12}+k_{14}+k_{1234}+k_{1432}-1}
%     \binom{k_{2}+k_{12}+k_{23}+k_{1234}+k_{1432}}{k_{12}+k_{23}+k_{1234}+k_{1432}}\\
%     &\binom{k_{3}+k_{23}+k_{34}+k_{1234}+k_{1432}-1}{k_{23}+k_{34}+k_{1234}+k_{1432}-1}
%     \binom{k_{4}+k_{14}+k_{34}+k_{1234}+k_{1432}-1}{k_{23}+k_{34}+k_{1234}+k_{1432}-1}\\
%     &\biggl(
%         \sum_{k_{23}^{12}+k_{23}^{14}=k_{14}} \sum_{k_{34}^{12}+k_{34}^{14}=k_{34}}
%         \binom{k_{23}^{12}+k_{12}+k_{1234}-1}{k_{23}^{12}}
%         \binom{k_{34}^{12}+k_{1234}+k_{23}^{12}-1}{k_{34}^{12}} \\
%         &\binom{k_{34}^{14}+k_{14}+k_{1432}-1}{k_{34}^{14}}
%         \binom{k_{23}^{14}+k_{34}^{14}+k_{1432}-1}{k_{23}^{14}} 
%     \biggr) + \\
% \end{align*}
\end{document}